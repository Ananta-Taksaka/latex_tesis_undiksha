%-----------------------------------------------------------------------------------------------%
%
% Maret 2019
% Template Latex untuk Tugas Akhir Program Studi Sistem informasi ini
% dikembangkan oleh Inggih Permana (inggihjava@gmail.com)
%
% Template ini dikembangkan dari template yang dibuat oleh Andreas Febrian (Fasilkom UI 2003).
%
% Orang yang cerdas adalah orang yang paling banyak mengingat kematian.
%
%-----------------------------------------------------------------------------------------------%

%-----------------------------------------------------------------------------%
\chapter{\babI}
%-----------------------------------------------------------------------------%

%-----------------------------------------------------------------------------%
\section{Latar Belakang}
%-----------------------------------------------------------------------------%
Meningkatnya kesadaran berqurban menjadikan peserta qurban bertambah setiap tahun \cite{mangalagowri2016eeg}. Peserta qurban bukan didominasi lagi oleh kalangan menengah ke atas, namun juga diikuti oleh masyarakat yang memprioritaskan dana yang dimiliki untuk berqurban. Selain itu, perekonomian yang semakin membaik \textit{juga} menjadi \textit{faktor meningkatnya} partisipasi masyarakat dalam berqurban.

Salah satu hewan yang sering dijadikan pilihan qurban adalah sapi. Sapi selalu menjadi langganan qurban di saat hari raya Idul Adha. Hal ini dapat dibuktikan pada setiap Mesjid yang terdapat di sekitar kota Pekanbaru, sapi menjadi mayoritas hewan yang akan disembelih pada hari raya Idul Adha dibandingkan hewan-hewan lainnya seperti kerbau dan kambing. Menurut \citeA{purnamasari2015aspects}, berdasarkan data dari tahun 2011 sampai 2014 menunjukkan bahwa dari 390 Mesjid yang terdapat di Pekanbaru rata-rata mengalami peningkatan penyembelihan sapi.


%-----------------------------------------------------------------------------%
\section{Perumusan Masalah}
%-----------------------------------------------------------------------------%
Perumusan masalah tugas akhir ini adalah bagaimana mengakuisisi kemampuan pakar-pakar yang berhubungan dengan kelayakan sapi qurban ke dalam sebuah basis pengetahuan dan mengimplementasikan basis pengetahuan tersebut ke dalam sistem berbasis mobile.


%-----------------------------------------------------------------------------%
\section{Batasan Masalah}
%-----------------------------------------------------------------------------%
Batasan masalah tugas akhir ini adalah:
\begin{enumerate}
	\item Pakar-pakar pada penelitian ini meliputi pakar peternakan, kesehatan hewan, dan syariah Islam.
	\item Sistem \textit{operasi} \textbf{smartphone} mobile yang digunakan adalah Android Lollipop.
	\item DST...
\end{enumerate}

%-----------------------------------------------------------------------------%
\section{Tujuan}
%-----------------------------------------------------------------------------%
Tujuan tugas akhir ini adalah:
\begin{enumerate}
	\item Tujuan 1.
	\item Tujuan 2.
	\item Tujuan 3.
	\item DST...
\end{enumerate}

%-----------------------------------------------------------------------------%
\section{Manfaat}
%-----------------------------------------------------------------------------%
Manfaat tugas akhir ini adalah:
\begin{enumerate}
	\item Tujuan 1.
	\item Tujuan 2.
	\item DST...
\end{enumerate}

%-----------------------------------------------------------------------------%
\section{Sistematika Penulisan}
%-----------------------------------------------------------------------------%
Sistematika penulisan laporan adalah sebagai berikut:

% \textbf{BAB 1. \babSatu}

% BAB 1 pada tugas akhir ini berisi tentang: (1) latar belakang masalah; (2) rumusan masalah; (3) batasan masalah; (4) tujuan; (5) manfaat; dan (6) sistematika penulisan.

% \textbf{BAB 2. \babDua}

% BAB 2 pada tugas akhir ini berisi tentang: DST.


% \textbf{BAB 3. \babTiga}

% BAB 3 pada tugas akhir ini berisi tentang: DST.


% \textbf{BAB 4. \babEmpat}

% BAB 4 pada tugas akhir ini berisi tentang: DST.

% \textbf{BAB 5. \babLima}

% BAB 5 pada tugas akhir ini berisi tentang: DST.

% \textbf{BAB 6. \babEnam}

% BAB 6 pada tugas akhir ini berisi tentang: DST.